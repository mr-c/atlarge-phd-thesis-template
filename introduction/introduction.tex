\chapter{Introduction}
\label{introduction}

\begin{abstract}
Sample Abstract.
\end{abstract}


\newpage

\dropcap{T}his is a introductory page.


%Introduction: 2-4 pages; look at existing group work; time-box write for 2 days; send for review & repeat.
%Context
%Problem statement
%Research questions — explicitly, also why they are scientific research questions
%Approach, including focus on methodology
%Contribution, explicitly and with details pointing to the chapters/sections where it appears in the thesis
%Conceptual
%Technical
%Community
%Dissemination
%Thesis reading guide
%Non-plagiarism statement – “I did this stuff, not ChatGPT or someone else”
%Open Science / Reproducibility statement


\section{Background \& Context}
In this thesis, you can reference pictures~\Cref{fig:devmodel} using Cleverref and circles \circled{5}.

\begin{figure}[htb]
	\centering
	\includegraphics[width=0.65\columnwidth]{development_model_without_papers}
	\caption{The stages of the FDD model and their relationship to other
          Software Engineering concepts.}
	\label{fig:devmodel}
\end{figure}

We also have lists:

\begin{enumerate}
  \item Static Analysis~\circled{3} examines program artifacts or
    their source code without executing them~%\cite{wichmann1995industrial}
		, while
 \item Dynamic Analysis~\circled{4} relies on information gathered from their
   execution~%\cite{cornelissen2009systematic}.
\end{enumerate}

Or boxes:

\begin{framed}
This thesis is concerned with the empirical assessment of the state of the art of how developers
drive software development with the help of feedback loops.
\end{framed}

Or code:
\begin{lstlisting}[caption={\textsc{TrinityCore}},label={lst:e1}]
 x += other.x;
 y += other.y;
 z += other.y;
\end{lstlisting}


I hope this helps you get started!
Moritz \& Laurens
